%%%%%%%%%%%%%%%%%%%%%%%%%%%%%%%%%%%%%%%%%%%%%%%%%
%---- LaTeX-Header fuer Abschlussarbeiten, Prof. Thomas Goerne, Dez. 2012/Aug. 2013 ----
%------ Modified by B.Sc. Julius Neudecker, May 2021 ------
%%%%%%%%%%%%%%%%%%%%%%%%%%%%%%%%%%%%%%%%%%%%%%%%%

\documentclass[11pt,paper=A4,pointlessnumbers,bibtotoc,liststotoc,DIV=13,BCOR=1mm]{scrreprt}
% BCOR ist die Bindekorrektur (verlorener Rand am linken Blattrand)! Wert haengt von der Art der Heftung ab!!
% DIV ist eine Satzspiegeleinstellung von KOMA-Script / sccreprt.

\pagestyle{headings}

\usepackage[T1]{fontenc} % Font Encoding fuer europaeische Schriften mit Umlauten (Unterstuetzung der Worttrennung)
\usepackage{lmodern} % PostScript-Varianten der TeX Computer Modern-Schriften laden
\usepackage[english]{babel} % Spracheinstellungen fuer Englisch und Neudeutsch laden

\usepackage{graphicx} % Grafikeinbindung (fuer .JPG, .JPEG, .PNG und .PDF, falls pdflatex benutzt wird)
\usepackage[table]{xcolor} % ermoeglicht farbige Schrift und farbige Tabellenzeilen
\definecolor{black}{gray}{0} % Umdefinition der Farbe black, falls noetig (0=schwarz, 1=weiss)
\definecolor{dblue}{rgb}{0.1,0.2,0.6} % Dunkelblau, fuer Hyperlinks
\definecolor{lgray}{gray}{0.9} % Hellgrau, fuer Tabellen (0=schwarz, 1=weiss)

\usepackage{booktabs} % fuer schoene Tabellen

\usepackage[round,authoryear]{natbib} % Literaturverweise mit Name/Jahreszahl in runden Klammern
\bibpunct[:\,]{(}{)}{,}{a}{}{,~}  % Feinformatierung der Natbib-Zitierweise

\usepackage[hyphens]{url}
\usepackage[colorlinks=true,linkcolor=black,citecolor=dblue,urlcolor=dblue]{hyperref} 
\usepackage{hyperref}  
% die Pakete url und hyperref ermoeglichen anklickbare URLs im Quellenverzeichnis in definierter Farbe, 
% sie ermoeglichen den Zeilenumbruch bei langen URLs, und sie erzeugen Hyperlinks (Farbe s.o.) 
% zwischen Quellenverweis und Quellenverzeichnis sowie zwischen label und ref im PDF-Dokument

%subfigues next to each other
\usepackage{subfig}

%too much whitespace otherwise
\usepackage[left=2cm,right=2cm,top=2cm,bottom=2cm]{geometry}

% Fonteinstellungen fuer Bildunterschriften: Unterschrift serifenlos, "Abbildung" fett (bfseries = bold face series)
\setkomafont{captionlabel}{\sffamily\bfseries}
\setkomafont{caption}{\sffamily}

% Ordner für Grafiken
\graphicspath{ {./images/} }

%% ToDo Notes
\newcommand\todo[1]{\textcolor{red}{#1}}

%------------------------------------------------------------------------------------------------------------------
%------ Eigenstaendigkeitserklaerung im gerahmten Kasten (parbox in einer framebox) ------
%------------------------------------------------------------------------------------------------------------------

\newcommand{\eigen}{
\setlength{\fboxsep}{2ex}
\setlength{\fboxrule}{0.8pt} 
% Einstellungen fuer Rahmenabstand und Rahmendicke der Framebox
\begin{center}
	\fbox{
		\parbox{0.8\linewidth}{
			I hereby confirm that this thesis is my own work and that I have not sought or used inadmissible help of third parties to produce this work and that I have clearly referenced all sources used in this thesis. I have fully referenced and used inverted commas for all text directly or indirectly quoted from a source.
		\par\bigskip\bigskip\bigskip\bigskip
		\hspace*{0.8cm}Place and date \hfill \vorname~\nachname\hspace*{0.8cm}
		}
	}
\end{center}
}

%%%%%%%%%%%%%%%%%%%%%%%%%%%%%%%%%%%%%%%%%%%%%%%%%