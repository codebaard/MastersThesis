\documentclass[titlepage, a4paper, 11pt]{scrartcl}

%too much whitespace otherwise
\usepackage[left=2cm,right=2cm,top=2cm,bottom=2cm]{geometry}

% Grafik Pakete
\usepackage{graphicx,hyperref,amssymb}

% Ordner für Grafiken
\graphicspath{ {./images/} }
\usepackage{float}

\usepackage[utf8]{inputenc}
\usepackage{amsmath}
\usepackage{amsfonts}
\usepackage{amssymb}
\usepackage{graphicx}

\usepackage{caption}
\usepackage{subcaption}

% Header and Footer
\usepackage{fancyhdr}

%bibtex
\usepackage{cite}
\usepackage{csquotes}

%code snippets
\usepackage{listings}
\usepackage{color}

\definecolor{dkgreen}{rgb}{0,0.6,0}
\definecolor{gray}{rgb}{0.5,0.5,0.5}
\definecolor{mauve}{rgb}{0.58,0,0.82}

\lstset{frame=tb,
    language=HTML,
    aboveskip=3mm,
    belowskip=3mm,
    showstringspaces=false,
    columns=flexible,
    basicstyle={\small\ttfamily},
    numbers=none,
    numberstyle=\tiny\color{gray},
    keywordstyle=\color{blue},
    commentstyle=\color{dkgreen},
    stringstyle=\color{mauve},
    breaklines=true,
    breakatwhitespace=true,
    tabsize=3
}

%dynamic image scaling
\newcommand{\imgScale}{0.7}


\pagestyle{fancy}
\fancyhf{}
\rhead{Julius Neudecker B.Sc.}
\lhead{Using a neural interface for interaction in Virtual Reality}
 
\begin{document}

    \title{Using a neural interface for interaction in Virtual Reality}
    \author{Julius Neudecker \\ Bachelor of Science \\ \href{mailto:julius.neudecker@haw-hamburg.de}{julius.neudecker@haw-hamburg.de} }
    \date{May 2021}

    \maketitle

    \tableofcontents

    \begin{abstract}
        Modern technology evolved to pick up the eletric signals emitted from the human brain in order to generate user input to eletronic equipment. This study aims to evaluate a demo use-case by using a neural interface from nextmind to control user interactions in Virtual Reality.
    \end{abstract}

    \section{Introduction}

        \subsection{Neural Interfaces}

            This section will be a general Introduction to elaborate the topic and putting it into context of the state of research

            \begin{itemize}
                \item Picking up brain activity
                \item Invasive vs non-invasive
                \item Usage for interacting with eletronic equipment
            \end{itemize}

        \subsection{Related work}

            Whats state of the art, what has been done so far in research and where is my study in context?

            \begin{itemize}
                \item State of research
                \item Applications in the medical domain
                \item Applications in the HCI domain
                \item other...
            \end{itemize}

        \subsection{Use case "Neural Interface in VR"}

            I don't have a certain use-case in mind at this stage. Therefore this section is still very generic at the moment.

            \begin{itemize}
                \item Use Case description
                \item Research goals
            \end{itemize}        

        \subsection{Hypothesis}

            \begin{itemize}
                \item definition of research goals
                \item hypothesis
            \end{itemize}
        
    \section{Technological challenges}

        Due to being non-invasive there must exist certain drawbacks with this technology. I want to examine the shortcomings and possible ways to overcome these.    
        A valuable resource of information might be nextminds homepage \cite{NextMind}.

        \subsection{Resolution of the Interface}

            \begin{itemize}
                \item definition of the resolution parameter
                \item input taxonomy diagram
                \item how to examine with survey                                     
            \end{itemize}

        \subsection{Constraints}

            As far as I understood, the interface allows for four different interaction goals. It would be interesting to see, which kinds of interaction are possible.                

            \begin{itemize}
                \item Interaction objects
                \item interaction types in regard to input taxonomy
                \item evaluation in user survey
            \end{itemize}
        
    \section{Survey Structure and layout}

        \subsection{Considerations}

            \begin{itemize}
                \item Which topics do I want to evaluate in detail
                \item what are my tools
                \item Who is my audience
                \item how to I operationalize the values for context
                \item What are my performance indicators
            \end{itemize}

        \subsection{Survey structure}
		
            Based on the findings, I want to define the survey in this section.

            \begin{itemize}
                \item item 1
                \item ...
            \end{itemize}

        \subsection{Survey}

            How is the survey carried out. This depends largely on the outcome of section survey structure.

            \begin{itemize}
                \item item 1
                \item ...
            \end{itemize}

    \section{Survey results}

        Once the study has been structured and carried out, I can write down the results.

    \section{Findings}

        This section also depends on the outcomes in context to the resarch question.

    \section{Conclusion}

        \subsection{Results}

            Summarizing the results and findings of the study briefly.

        \subsection{Future Work}

            Based on the findings and new devices on the horizon, this should give a brief outlook on how to continue this research.

        \subsection{Acknowledgements}

            ...

    \bibliography{references.bib} 
    \bibliographystyle{ieeetr}

\end{document}