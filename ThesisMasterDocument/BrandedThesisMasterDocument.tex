%%%%%%%%%%%%%%%%%%%%%%%%%%%%%%%%%%%%%%%%%%%%%%%%%
%------ LaTeX-Template für Abschlussarbeiten, Prof. Thomas Görne, Dezember 2012 --------
%------ Modified by B.Sc. Julius Neudecker, May 2021 ------
%%%%%%%%%%%%%%%%%%%%%%%%%%%%%%%%%%%%%%%%%%%%%%%%%

%---- Header (mit Formateinstellugen) laden, Inputencoding prxfen ------

%%%%%%%%%%%%%%%%%%%%%%%%%%%%%%%%%%%%%%%%%%%%%%%%%
%---- LaTeX-Header fuer Abschlussarbeiten, Prof. Thomas Goerne, Dez. 2012/Aug. 2013 ----
%------ Modified by B.Sc. Julius Neudecker, May 2021 ------
%%%%%%%%%%%%%%%%%%%%%%%%%%%%%%%%%%%%%%%%%%%%%%%%%

\documentclass[11pt,paper=A4,pointlessnumbers,bibtotoc,liststotoc,DIV=13,BCOR=1mm]{scrreprt}
% BCOR ist die Bindekorrektur (verlorener Rand am linken Blattrand)! Wert haengt von der Art der Heftung ab!!
% DIV ist eine Satzspiegeleinstellung von KOMA-Script / sccreprt.

\pagestyle{headings}

%Zeilenabstand
\usepackage{setspace}
\onehalfspacing

\usepackage[T1]{fontenc} % Font Encoding fuer europaeische Schriften mit Umlauten (Unterstuetzung der Worttrennung)
\usepackage{lmodern} % PostScript-Varianten der TeX Computer Modern-Schriften laden
\usepackage[english]{babel} % Spracheinstellungen fuer Englisch und Neudeutsch laden

\usepackage{graphicx} % Grafikeinbindung (fuer .JPG, .JPEG, .PNG und .PDF, falls pdflatex benutzt wird)
\usepackage[table]{xcolor} % ermoeglicht farbige Schrift und farbige Tabellenzeilen
\definecolor{black}{gray}{0} % Umdefinition der Farbe black, falls noetig (0=schwarz, 1=weiss)
\definecolor{dblue}{rgb}{0.1,0.2,0.6} % Dunkelblau, fuer Hyperlinks
\definecolor{lgray}{gray}{0.9} % Hellgrau, fuer Tabellen (0=schwarz, 1=weiss)

\usepackage{booktabs} % fuer schoene Tabellen
\usepackage{enumitem} % for letter enumerations

\usepackage[round,authoryear]{natbib} % Literaturverweise mit Name/Jahreszahl in runden Klammern
\bibpunct[:\,]{(}{)}{,}{a}{}{,~}  % Feinformatierung der Natbib-Zitierweise

\usepackage[hyphens]{url}
\usepackage[colorlinks=true,linkcolor=black,citecolor=dblue,urlcolor=dblue]{hyperref} 
\usepackage{hyperref}  
% die Pakete url und hyperref ermoeglichen anklickbare URLs im Quellenverzeichnis in definierter Farbe, 
% sie ermoeglichen den Zeilenumbruch bei langen URLs, und sie erzeugen Hyperlinks (Farbe s.o.) 
% zwischen Quellenverweis und Quellenverzeichnis sowie zwischen label und ref im PDF-Dokument

%subfigues next to each other
\usepackage{subfig}

%too much whitespace otherwise
\usepackage[left=2.5cm,right=2.5cm,top=2.5cm,bottom=2.5cm]{geometry}

%include other documents
\usepackage{standalone}
%\usepackage[nogin]{Sweave}
\usepackage{pdfpages}

%for logfiles
\usepackage{listings}
\usepackage{color}

\definecolor{dkgreen}{rgb}{0,0.6,0}
\definecolor{gray}{rgb}{0.5,0.5,0.5}
\definecolor{mauve}{rgb}{0.58,0,0.82}

\lstset{frame=tb,
    language=HTML,
    aboveskip=3mm,
    belowskip=3mm,
    showstringspaces=false,
    columns=flexible,
    basicstyle={\small\ttfamily},
    numbers=none,
    numberstyle=\tiny\color{gray},
    keywordstyle=\color{blue},
    commentstyle=\color{dkgreen},
    stringstyle=\color{mauve},
    breaklines=true,
    breakatwhitespace=true,
    tabsize=3
}

% Fonteinstellungen fuer Bildunterschriften: Unterschrift serifenlos, "Abbildung" fett (bfseries = bold face series)
\setkomafont{captionlabel}{\sffamily\bfseries}
\setkomafont{caption}{\sffamily}

% Ordner für Grafiken
\graphicspath{ {./images/} }

%% ToDo Notes
\newcommand\todo[1]{\textcolor{red}{#1}}

%------------------------------------------------------------------------------------------------------------------
%------ Eigenstaendigkeitserklaerung im gerahmten Kasten (parbox in einer framebox) ------
%------------------------------------------------------------------------------------------------------------------

\newcommand{\eigen}{
\setlength{\fboxsep}{2ex}
\setlength{\fboxrule}{0.8pt} 
% Einstellungen fuer Rahmenabstand und Rahmendicke der Framebox
\begin{center}
	\fbox{
		\parbox{0.8\linewidth}{
			I hereby confirm that this thesis is my own work and that I have not sought or used inadmissible help of third parties to produce this work and that I have clearly referenced all sources used in this thesis. I have fully referenced and used inverted commas for all text directly or indirectly quoted from a source.
		\par\bigskip\bigskip\bigskip\bigskip
		\hspace*{0.8cm}Place and date \hfill \vorname~\nachname\hspace*{0.8cm}
		}
	}
\end{center}
}

%%%%%%%%%%%%%%%%%%%%%%%%%%%%%%%%%%%%%%%%%%%%%%%%%

\usepackage[utf8]{inputenc} % Inputencoding, universell

%------------------------ Titelblatt-Layout laden ----------------------------------

%%%%%%%%%%%%%%%%%%%%%%%%%%%%%%%%%%%%%%%%%%%%%%%%%
%------ LaTeX-Titelblatt fuer Bachelorarbeiten, Prof. Thomas Goerne, Dezember 2012 -------
%------------------------------------------------------------------------------------------------------------------
%--------------------------------- Deklarationen fuer die Titelseite  --------------------------------------
%%%%%%%%%%%%%%%%%%%%%%%%%%%%%%%%%%%%%%%%%%%%%%%%%

\title{\titel\\[2ex]
\LARGE Masters Thesis\\
\large To obtain the academic degree M.Sc.\\[1.5ex]
\LARGE \vorname~\nachname\\[0.5ex] 
\large \matrikelnummer
}

\author{\unitlength1mm
\large\raisebox{-1ex}{\includegraphics[width=4em]{HAW_wuerfel}}\hspace{1ex}
\parbox[b]{11.2cm}{\sffamily\large%
University of applied sciences Hamburg\\[-0.2ex]
Faculty of Design, Media und Information\\[-0.2ex]
Department of Media Engineering
}\\[6ex]
\sffamily\large First examiner: \erstpruef\\[0.5ex]
\sffamily\large Second examiner: \zweitpruef}

%%%%%%%%%%%%%%%%%%%%%%%%%%%%%%%%%%%%%%%%%%%%%%%%%

%---------------------------- Titeldefinitionen --------------------------------------

\newcommand{\vorname}{Julius}
\newcommand{\nachname}{Neudecker}
\newcommand{\matrikelnummer}{2025850}

\newcommand{\titel}{{[Working Title] Using a neural interface for interaction in virtual reality}\\[0.2ex] 
				\Large an HCI study}

\newcommand{\erstpruef}{Prof. Dr.Roland Greule}
\newcommand{\zweitpruef}{Dipl. Inf. Rüdiger Höfert}

\date{preliminary version from \today}   % praktisch fxr Vorab-Versionen. 
%\date{\sffamily Hamburg, 30.06.2021}  % Abgabedatum!

%--------------------------------------------------------------------------------------
%----------------------------- hier gehts los! --------------------------------------
%--------------------------------------------------------------------------------------

\begin{document}
    \selectlanguage{english}
    \maketitle
    \tableofcontents
    \clearpage          % Seitenumbruch

    %------------ Zusammenfassung / Abstract ------------------

    \thispagestyle{empty}
    \selectlanguage{english}
    \section*{\centering\abstractname}
    Modern technology evolved to pick up the eletric signals emitted from the human brain in order to generate user input to eletronic equipment. This study aims to evaluate a demo use-case by using a neural interface from nextmind to control user interactions in Virtual Reality.

    % --- Julius Text Sections ---

    \chapter{Introduction}

        %what is it
        %why does it make sense
        %what is the market evaluation


        In recent years significant progress has been made on the development of interfaces which relies on direct interaction with the brain itself. \todo{find some sources} The latest popular example is \textit{Neuralink} with their monkey learning to play the game \textit{Pong} only by using its brain (\cite{Neuralink.2021}). However there are more examples of a working interface: \todo{Do som research here}.
        These interfaces are generally called \textit{Brain-Computer-Interface} or \textit{BCI} in short. The general working principle is sensing the electrical signals of the brain and use this information to generate any kind of arbitrary output \todo{find source}. 
        In certain use cases like motoric reactions to visual cues, this could potentially reduce total reaction time. This study aims to examine a potential use case with a device which is readily available to consumers.
        

        \section{Brain-Computer-Interfaces}\label{intro-bci}

            In this section a general overview of the working principle of these interfaces will be provided. Since this study is aimed at computer science and HCI\footnote{Human Computer Interaction}, the neuroscience and medical domain will be only covered very briefly.

            %https://www.frontiersin.org/articles/10.3389/fnins.2016.00295/full

            First studies began by \cite{Vidal.1973}, who investigated the possibility to use EEG\footnote{Electroencephalogram} waves, which were first recorded by \cite{Berger.1929}, as a way to create a direct interaction between a machine and a human brain. 

            There are three types of BCIs: invasive, partially invasive and non invasive. This depicts the degree of intrusion into the skull and brain tissue. \textit{Invasive} BCIs are electrodes, which are implanted directly into or onto the grey matter of the brain. This can cause long term issues like scars and also degraded singal strength according to \cite{Abdulkader.2015}. 
            Partially invasive BCI however are although located within the skull not in direct contact with the grey matter.
            Non-Invasive BCI are only placed on the head without intrusion of any tissue.
            Due to the direct contact, invasive BCI provide the best resolution of the measured signals. Non-invasive BCI in comparison suffer from signal degradation and deformation of the cranial bone tissue. 
            Therefore partially invasive BCI are a compromise between good signal strength and the risk of medical conditions.
            Another potential advantage of non-invasive BCIs is that these Interfaces could be easier mass-produced and become affordable to consumers. Also they don't require specialized medical knowledge and equipment to operate.

            The way these interfaces work is based on the same principle: A human brain emits electrical signals, which can be picked up.
            According to \cite{Vidal.1973}, they can be described as follows:

            \medskip
            \emph{"Embedded in this sustained "spontaneous" or "ongoing" electrical activity, short, distinctive (0.5-2 sec) waveforms can be found that are evoked, for instance, when a brief sensory message (stimulus) such as a brief illumination of the visual field or a tap on the forearm is received by the subject."}
            \medskip

            Based on the origin within the brain, these can be correlated to certain stimuli, mental and emotional states (\cite{JardimGoncalves.2018}) and according to \cite{Waldert.2016} been used to drive \emph{an external effector or affecting internal body parts and functions.} The external effector is the use case which is being examined in this study.

            Without a BCI, interaction with a computer requires some physical interaction with devices such as keyboards, mouses or gestures on a touch screen. There are mainly two different reasons, why these devices are a constraint to speed and efficiency of HCI. The first reason is a limitation on interaction speed: Although there is no definitive concensus about the speed of thinking, the \todo{majority of people would agree}, that typing along in the same speed as the thinking process is usually impossible - except for rare individuals who devoted a significant time practicing. Secondly: in applications such as games, where reaction time and accuracy is the fundamental element for success or failure, an interaction based on motoric interaction with a physical pointing device has some significant drawbacks like limited accuracy, if the whole chain of wrist movement in conjunction with a mouse is under scrutiny. 

            If a BCI was to replace these types interaction, these constraints could potentially be alleviated and interaction based on physical interaction rendered obsolete. 

        \section{Working principle}            

            Before any deeper considerations in regard to the general scope of this study can be made, it is important to understand the working principle of the BCI, which will be used. Although the vendor of the BCI in question does not disclose any details of the inner workings itself, it is safe to assume that the underlying technique used is the so called \textit{Visually Evoked Potential} - VEP in short. \cite{Sokol.1976} provides detailed inside into the topic from a neuroscientific point of view. The general principle however is that any visual stimuli cause a certain pattern of waves within the visual cortex of the brain. These patterns can be used to evaluate if a certain pattern is being seen \textit{and} in fokus of the person. This is being done by subsequently feeding the sensor data through a trained neural network. The objects, which are being seen by the person, have been labeled \textit{neurotags} (\cite{NextMind.23112020}) from the vendor of the BCI. These neurotags can provide two different readouts: If it is triggered (i.E. \textit{seen}) and the confidence, which depicts the level of \textit{focus} of the user on the neurotag (\cite{NextMind.18112020}).
            
            \begin{figure}[h]     % h=here, t=top, b=bottom, p=page
                \centering
                \includegraphics[width=0.8\textwidth]{electrodes-sensor} 
                \caption{Physical layout of the sensor}\label{electrodes-sensor}
            \end{figure}
            
            The physical layout of the sensor is show at figure \ref*{electrodes-sensor}. It has 18 eletrodes, which are arragend in pairs to cover the area, where the visual cortex is located at on the back of the cranium. It is battery driven and communicates via the Bluetooth LowEnergy protocol.

        \section{Related work}

            Non-Invasive BCI based on VEP has been around for while now and therefore a significant number of studies have been carried out to further examine their performance, advance development and evaluate use-cases. To keep track of the recent advancements, it is best to consider recent conferences for the most recent resarch.

            %conferences:
            % https://www.aconf.org/conf_177761.html
            % https://bcisociety.org/bci-meeting/
            % 

            \begin{itemize}
                \item State of research
                \item Applications in the HCI domain
                \item other...
            \end{itemize}

            \todo{This chapter is very much WIP}

        \section{Use case "Neural Interface in VR"}

            Before any use case can be conceived, it has to be determined what kind of interaction this interface allows. Section \ref*{intro-bci} covered briefly the concept of input taxonomies to elaborate optimization potentials with BCIs. According to \todo{Find some source} an input taxonomy depicts the DOF\footnote{Dimensions Of Freedom} and the granularity and magnitudes in regard to the interface which this interface offers.

            \begin{figure}[h]     % h=here, t=top, b=bottom, p=page
                \centering
                \includegraphics[width=0.8\textwidth]{placeholder.jpg} 
                \caption{Taxonomy of the nextmind BCI. \todo{Todo: Create taxo schema}}\label{bci-taxonomy}
            \end{figure}

            Figure \ref*{bci-taxonomy} shows the input taxonomy of the BCI in question. It is derived from the API\footnote{Application Programming Interface} endpoints which the the SDK\footnote{Software Development Kit} of the sensor offers: \cite{NextMind.18112020}
            . The only two \textit{tracking resuls} are \textit{hit} and \textit{confidence}. Where hit is a two state interaction: the neurotag is being seen by the user and subsequently recognized by the sensor and its backend or it is not. The confidence property depicts the attention which the user is paying to the \textit{neurotag}. This is a continuous decimal value between 0 and 1.
            The fact that these types of interaction are based on neural activity raises the question if a pure mapping of continuous and discrete input modalities to established interfaces would be beneficial to the user experience. Under the reasonable assumption that without any training the metric \textit{focus} can only be deliberately controlled on a very coarse level, the necessary sensitiveness required for modern GUIs can not be achieved with this particular sensor. The remaining two state property, which can be utilized to select or deselect certain objects also only allows for limited interaction. However, these neurotags can be placed in arbitrary places. Although a \textit{toggle}-like behavior is not mentioned explicitly, it might be possible to de-select any activated neurotag when the \textit{focus} property falls under a certain value.

            \bigskip

            Based on the previous reasoning, the following questions can be raised in regard to the feasibility of any interface which could potententially be conceived with this technology:

            \begin{itemize}
                \item How fast is the perceived and measured reaction time of these neurotags?
                \item What is the minimum size the neurotags have to have in order to be recognizable?
                \item Is the interface usable for brains of all ages or do gerontological effects have an effect on usability?
                \item Do certain medical conditions (i.e. attentiveness disorder) have an impact on the usability?
                \item How fast can a user switch between neurotags?
                \item Is a BCI controlled GUI intuitive to use?
                \item Does a personal affinity to technology have an influence on the perceived difficulty of interaction?
            \end{itemize}

            These questions can be clustered into two groups: \textit{neurological} and \textit{interaction}. Although these considerations open up a vast space of potential cases. Therefore the priority is to examine wheter these interfaces are generally usable by the majority of users and if these interfaces are intuitive to use. 

        \section{Hypothesis}

            The considerations in the previous section leads to these two hypothesis:

            \medskip
            \emph{"Age does not have a detrimental effect on the ability to use a non-invasive BCI based on VEP technology."}
            \medskip

            and

            \medskip
            \emph{"VEP BCI operated GUIs are intuitive to use."}
            \medskip

    \chapter{Technological challenges}

        Due to being non-invasive there must exist certain drawbacks with this technology. I want to examine the shortcomings and possible ways to overcome these.    
        A valuable resource of information might be nextminds homepage.

        \section{Resolution of the Interface}

            \begin{itemize}
                \item definition of the resolution parameter
                \item input taxonomy diagram
                \item how to examine with survey                                     
            \end{itemize}

        \section{Constraints}

            As far as I understood, the interface allows for four different interaction goals. It would be interesting to see, which kinds of interaction are possible.                

            \begin{itemize}
                \item Interaction objects
                \item interaction types in regard to input taxonomy
                \item evaluation in user survey
            \end{itemize}

    \chapter{Survey Structure and layout}

        \section{Considerations}

            \begin{itemize}
                \item Which topics do I want to evaluate in detail
                \item what are my tools
                \item Who is my audience
                \item how to I operationalize the values for context
                \item What are my performance indicators
            \end{itemize}

        \section{Survey structure}

            Based on the findings, I want to define the survey in this section.

            \begin{itemize}
                \item item 1
                \item ...
            \end{itemize}

        \section{Survey}

            How is the survey carried out. This depends largely on the outcome of section survey structure.

            \begin{itemize}
                \item item 1
                \item ...
            \end{itemize}

    \chapter{Survey results}

        Once the study has been structured and carried out, I can write down the results.

    \chapter{Findings}

        This section also depends on the outcomes in context to the resarch question.

    \chapter{Conclusion}

        \section{Results}

            Summarizing the results and findings of the study briefly.

        \section{Future Work}

            Based on the findings and new devices on the horizon, this should give a brief outlook on how to continue this research.

    \chapter{Acknowledgements}

        ...

    \appendix

        \chapter{Material}

            \section{Surveys, Protocols, etc.}

                Neque porro quisquam est qui dolorem ipsum quia dolor sit amet, consectetur, adipisci velit...

        %--------------------- VERZEICHNISSE ----------------

        \listoffigures % Abbildungsverzeichnis erzeugen
        \listoftables % Tabellenverzeichnis erzeugen

    %--------------------- LITERATURLISTE ---------------
    % Die Einträge sollen alphabetisch sortiert sein.

    \bibliography{references} 
    \bibliographystyle{unsrtnat}

    %--------------------- EIGENSTÄNDIGKEITSERKLÄRUNG ---------------
    \clearpage\thispagestyle{empty}
    \eigen  % im header definiert
    %--------------------------------------- ENDE ------------------------------------

\end{document}