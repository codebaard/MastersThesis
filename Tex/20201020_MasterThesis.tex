\documentclass[titlepage, a4paper, 11pt]{scrartcl}

%too much whitespace otherwise
\usepackage[left=30mm,top=26mm,right=26mm,bottom=15mm]{geometry}

% deutsche Übersetzungen
%\usepackage[ngerman]{babel}
% Grafik Pakete
\usepackage{graphicx,hyperref,amssymb}
% Ordner für Grafiken
\graphicspath{ {./images/} }
% Pakete für Formatierung der Grafiken
\usepackage{wrapfig}
\usepackage{float}
% deutsches Encoding (Umlaute)
\usepackage[utf8]{inputenc}
% für Grad Symbol
\usepackage{textcomp}


%image grid
\usepackage{graphicx}
\usepackage{subfig}

%\usepackage{multicol}

% Header and Footer
\usepackage{fancyhdr}


\pagestyle{fancy}
\fancyhf{}
\rhead{Julius Neudecker B.Sc.}
\lhead{HHarbour Digital Twin - a serverside rendering usecase}
 
\begin{document}

    \title{HHarbour Digital Twin - a serverside rendering usecase}
    \author{Julius Neudecker \\ Bachelor of Science \\ \href{mailto:julius.neudecker@haw-hamburg.de}{julius.neudecker@haw-hamburg.de} }
    \date{Novembre 2020}

    \maketitle

    \tableofcontents

    \begin{abstract}
        Using serverside rendering to create an immersive experience on mobile devices without hardware constraints
        by rendering in the cloud and stream content and interaction.
    \end{abstract}

    \section{Introduction}

        \subsection{Problem domain}

            \begin{itemize}
                \item Mobile AR Graphics
                \item Tradeoff between performance and mobility
            \end{itemize}

        \subsection{State of technology - 4th quarter 2020}

            \begin{itemize}
                \item Gardner Hype Cyclce
                \item AR in Commercial Projects
                \item "AR Is useless" -> Techlead Quote
            \end{itemize}

        \subsection{Use Case "Digital Twin" HHarbour}

            Since this is a commercial work, the commercial aspect should be mentioned here and dicussed briefly at least.

            \begin{itemize}
                \item Use Case description
                \item Business Value
                \item Stakeholder
            \end{itemize}
        
        
        \subsection{Technological challenges}

            \textbf{This whole section might be better in the implementation section. However I'll keep it here for now to discuss topics.}

            \subsubsection{Mobile device constraints}

                \begin{itemize}
                    \item Battery vs. Performance
                    \item Heat
                    \item ARM Chips vs. Rendering                    
                \end{itemize}

            \subsubsection{Network}

                This is especially important for user experience. 

                I Kept this Section. But more for the sake of disussing the scope of the problem domain and some research but not 
                with the intention to make this a part of this research.

                \begin{itemize}
                    \item 5G Testsite at HH Harbour
                    \item WIFI 6
                    \item Software defined Network
                \end{itemize}

            \subsubsection{Bandwidth}

                Discuss this and find solutions to mitigate

            \subsubsection{Packet loss}

                Is this really a problem?
                Can this be mitigated?

            \subsubsection{Server ressources}

                Discuss this:

                \begin{itemize}
                    \item which ressources are really needed 
                    \item Kubernetes Cluster
                    \item \dots
                \end{itemize}
        
        \section{Related Work}

                \begin{itemize}
                    \item Nvidia and Microsoft Cloud Gaming
                    \item XRchitecture
                    \item \dots
                \end{itemize}

        \subsection{Hypothesis}

            Underlying questions:

            \begin{itemize}
                \item Is it possible from a UX-Standpoint to create an immersive AR-experience with SSR?
                \item \dots
            \end{itemize}

    \section{Architecture Backend}

        \subsection{Considerations}

            \begin{itemize}
                \item Off-Screen Rendering (Render to texture -> Base64 stream)
                \item Sessions and Access Tokens (Photon Network?)
                \item 
            \end{itemize}

        \subsection{How to lightweight and fast}
		
			\subsubsection{nVIDIA Maxine}
				
				Encode videostreams with ML to save up to 90\% bandwith with equal results.

        \subsection{Loadbalancing and Containerization}

            \begin{itemize}
                \item Kubernetes Cluser
                \item Every User-Session is dedicated container
            \end{itemize}

    \section{Architecture Fronend}

        \subsection{Software, SDK, Unity}

            Discuss topics such as:

            \begin{itemize}
                \item Cross Platform vs. Ecosystems
                \item External Libraries
                \item \dots
                \item Rapid changing Hard- and Software environments
            \end{itemize}

        \subsection{UI/UX}

            \begin{itemize}
                \item User Story Considerations
                \item Wireframing
                \item Klickdummy
                \item Create Intriguing Design
            \end{itemize}

    \section{Implementation}

        \subsection{Code Domains / Work Items}
            \begin{itemize}
                \item Serverside: Session Handler, Rendercluster, Nvidia Cloud XR
                \item Communication Layer: Session Tokens, Data Streams
                \item Frontend: UI, Render View to Screen, Interaction
            \end{itemize}
        
        \subsection{Milestones}
        
            Probably nice to have a milestone map and reflect this in restrospect.

    \section{Evaluation and research}

        \subsection{Define research goals}

        \subsection{User surveys}

            In order to evaluate UX, HCI aspects have to be considered.

        \subsection{Technical analysis}

    \section{Conclusion}

        \subsection{Results}

        \subsection{Future Work}

        \subsection{Acknowledgements}

    \bibliography{updatedReferences} 
    \bibliographystyle{ieeetr}


\end{document}